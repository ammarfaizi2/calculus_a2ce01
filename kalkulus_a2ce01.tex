\documentclass[13pt]{article}
\usepackage{amsmath}
\usepackage{amssymb}
\usepackage{amsfonts}
\usepackage{nccmath}
\usepackage{cancel}
\usepackage{color}
\usepackage{xcolor}
\usepackage{graphicx}
\usepackage{tikz}
\usepackage[margin=0.2in]{geometry}
\usepackage[utf8]{inputenc}
\thispagestyle{empty}
\begin{document}
\par\noindent
\begin{fleqn}[4em]

% No.1
\begin{align*}
\boxed{\begin{aligned}
  & \textbf{No. 1} \text{ Diketahui }
    f(x) = \frac{1}{4} x^4 - \frac{2}{3} x^3 - \frac{1}{2} x^2 + 2x - 1 \\
  & \text{a. Nilai } x \text{ yang memberikan titik kritis.} \\
  & f'(x) = \frac{1 \cdot 4}{4} x^3 - \frac{2 \cdot 3}{3} x^2 - \frac{1 \cdot 2}{2} x + 2 \\
  & f'(x) = \frac{1 \cdot \cancel{4}}{\cancel{4}} x^3 - \frac{2 \cdot \cancel{3}}{\cancel{3}} x^2 - \frac{1 \cdot \cancel{2}}{\cancel{2}} x + 2 \\
  & f'(x) = x^3 - 2x^2 - x + 2 \\
  & \text{Titik kritis terdapat pada } f'(x) = 0 \text{ atau } f'(x) \text{ tidak terdefinisi.} \\
  & f'(x) \text{ terdefinisi untuk semua nilai } x \\
  & \text{Cek } f'(x) = 0 \\
  & f'(x) = x^3 - 2x^2 - x + 2 = 0 \\
  & x^3 - 2x^2 - x + 2 = 0 \\
  & (x - 2)(x + 1)(x - 1) = 0 \\
  & x = -1, x = 2, x = 1 \\
  & \boxed{\therefore \text{Nilai } x \text{ yang memberikan titik kritis adalah } -1, 2 \text{ dan } 1} \\
  & \boxed{\begin{aligned}
    & f(x) \text{ naik jika } f'(x) > 0 \\
    \end{aligned}}\;~\;~\;~
    \boxed{\begin{aligned}
    & f(x) \text{ turun jika } f'(x) < 0 \\
    \end{aligned}}
  \\ & \textbf{by Ammar Faizi}
\end{aligned}}
\end{align*}

\end{fleqn}
\end{document}
