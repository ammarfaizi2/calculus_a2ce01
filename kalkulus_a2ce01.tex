\documentclass[13pt]{article}
\usepackage{amsmath}
\usepackage{amssymb}
\usepackage{amsfonts}
\usepackage{nccmath}
\usepackage{cancel}
\usepackage{color}
\usepackage{xcolor}
\usepackage{graphicx}
\usepackage{tikz}
\usepackage[margin=0.2in]{geometry}
\usepackage[utf8]{inputenc}
\thispagestyle{empty}
\begin{document}
\par\noindent
\begin{fleqn}[4em]

% No.1
\begin{align*}
\boxed{\begin{aligned}
  & \textbf{No. 1. } \text{ Diketahui }
    f(x) = \frac{1}{4} x^4 - \frac{2}{3} x^3 - \frac{1}{2} x^2 + 2x - 1 \\
  & \boxed{\begin{aligned}
    & f'(x) = \frac{1 \cdot 4}{4} x^3 - \frac{2 \cdot 3}{3} x^2 - \frac{1 \cdot 2}{2} x + 2 \\
    & f'(x) = \frac{1 \cdot \cancel{4}}{\cancel{4}} x^3 - \frac{2 \cdot \cancel{3}}{\cancel{3}} x^2 - \frac{1 \cdot \cancel{2}}{\cancel{2}} x + 2 \\
    & f'(x) = x^3 - 2x^2 - x + 2 \\
  \end{aligned}}\;~\;
  \boxed{\begin{aligned}[t]
    & f''(x) = 3x^2 - 4x - 1
  \end{aligned}}
  & \textbf{a.} \text{ Nilai } x \text{ yang memberikan titik kritis.} \\
  & \text{Titik kritis terdapat pada } f'(x) = 0 \text{ atau } f'(x) \text{ tidak terdefinisi.} \\
  & f'(x) \text{ terdefinisi untuk semua nilai } x \\
  & \text{Cek } f'(x) = 0 \\
  & f'(x) = x^3 - 2x^2 - x + 2 = 0 \\
  & x^3 - 2x^2 - x + 2 = 0 \\
  & (x - 2)(x + 1)(x - 1) = 0 \\
  & x = -1, x = 2, x = 1 \\
  & \boxed{\therefore \text{Nilai } x \text{ yang memberikan titik kritis adalah } \left\{-1, 2, 1\right\}} \\
  & \textbf{b.} \text{ Menentukan di mana } f(x) \text{ naik dan } f(x) \text{ turun.} \\
  & \boxed{\begin{aligned}
    & {\color{purple}(1)}\; f(x) \text{ naik jika } f'(x) > 0 \\
    & {\color{purple}(2)}\; x^3 - 2x^2 - x + 2 > 0 \\
    & {\color{purple}(3)}\; (x - 2)(x + 1)(x - 1) > 0 \\
    & {\color{purple}(4)}\; -1<x<1\text{ atau }x>2 \\
    & {\color{purple}(5)}\; \boxed{
      \therefore \text{Jadi, fungsi naik pada interval } (-1,\:1)\cup (2,\:\infty)
    }
  \end{aligned}}\;~\;
    \boxed{\begin{aligned}
    & {\color{purple}(1)}\; f(x) \text{ turun jika } f'(x) < 0 \\
    & {\color{purple}(2)}\; x^3 - 2x^2 - x + 2 < 0 \\
    & {\color{purple}(3)}\; (x - 2)(x + 1)(x - 1) < 0 \\
    & {\color{purple}(4)}\; x<-1\text{ atau }1<x<2 \\
    & {\color{purple}(5)}\; \boxed{
      \therefore \text{Jadi, fungsi turun pada interval } (-\infty,\:-1)\cup (1,\:2)
    }
    \end{aligned}} \\
  & \textbf{c.} \text{ Menentukan di mana } f(x) \text{ cembung ke atas } f(x) \text{ cekung ke bawah.} \\
  & \boxed{\begin{aligned}
    & {\color{purple}(1)}\; f(x) \text{ cembung ke atas jika } f''(x) > 0 \\
    & {\color{purple}(2)}\; x^3 - 2x^2 - x + 2 > 0 \\
    & {\color{purple}(3)}\; (x - 2)(x + 1)(x - 1) > 0 \\
    & {\color{purple}(4)}\; -1<x<1\text{ atau }x>2 \\
    & {\color{purple}(5)}\; \boxed{
      \therefore \text{Jadi, fungsi naik pada interval } (-1,\:1)\cup (2,\:\infty)
    }
  \end{aligned}}\;~\;
    \boxed{\begin{aligned}
    \end{aligned}} \\
  \\ & \textbf{by Ammar Faizi}
\end{aligned}}
\end{align*}

\end{fleqn}
\end{document}
